\documentclass{beamer}
\usepackage{ctex}
\usepackage{minted}
\usepackage{smartdiagram}
\usetikzlibrary{positioning}
\usetheme{metropolis}           % Use metropolis theme
\title{Flink中的状态管理}
\date{\today}
\author{左元}
\institute{尚硅谷 大数据组}
\begin{document}
  \maketitle
  \begin{frame}
    \frametitle{主要内容}

    \begin{itemize}
        \item Flink中的状态
        \item 算子状态(Operator State)
        \item 键控状态(Keyed State)
        \item 状态后端(State Backends)
    \end{itemize}
  
  \end{frame}

  \begin{frame}
      \frametitle{Flink中的状态}

      \begin{figure}
        \centering
        \includegraphics[width=0.6\textwidth]{image30.png}
        \caption{Flink中的状态}
      \end{figure}
  
      \begin{itemize}
          \item 由一个任务维护,并且用来计算某个结果的所有数据,都属于这个任务的状态
          \item 可以认为状态就是一个本地变量,可以被任务的业务逻辑访问
          \item Flink会进行状态管理,包括状态一致性、故障处理以及高效存储和访问,以便开发人员可以专注于应用程序的逻辑
      \end{itemize}
  
  \end{frame}

  \begin{frame}
      \frametitle{Flink中的状态}
  
      \begin{itemize}
          \item 在Flink中,状态始终与特定算子相关联
          \item 为了使运行时的Flink了解算子的状态,算子需要预先注册其状态
      \end{itemize}

      \textbf{总的说来,有两种类型的状态:}

      \begin{itemize}
          \item 算子状态(Operator State):算子状态的作用范围限定为算子任务
          \item 键控状态(Keyed State):根据输入数据流中定义的键(key)来维护和访问
      \end{itemize}
  
  \end{frame}

  \begin{frame}
      \frametitle{算子状态}

      \begin{figure}
        \centering
        \includegraphics[height=0.4\textheight]{image31.png}
        \caption{算子状态}
      \end{figure}
  
      \begin{itemize}
          \item 算子状态的作用范围限定为算子任务,由同一并行任务所处理的所有数据都可以访问到相同的状态
          \item 状态对于同一任务而言是共享的
          \item 算子状态不能由相同或不同算子的另一个任务访问
      \end{itemize}
  
  \end{frame}

  \begin{frame}
      \frametitle{键控状态(Keyed State)}

      \begin{figure}
          \centering
          \includegraphics[height=0.3\textheight]{image32.png}
          \caption{键控状态}
      \end{figure}
  
      \begin{itemize}
          \item 键控状态是根据输入数据流中定义的键(key)来维护和访问的
          \item Flink为每个key维护一个状态实例,并将具有相同键的所有数据,都分区到同一个算子任务中,这个任务会维护和处理这个key对应的状态
          \item 当任务处理一条数据时,它会自动将状态的访问范围限定为当前数据的key
      \end{itemize}
  
  \end{frame}

  \begin{frame}
      \frametitle{键控状态数据结构}
  
      \begin{itemize}
          \item 值状态(ValueState):将状态表示为单个的值
          \item 列表状态(List State):将状态表示为一组数据的列表
          \item 字典状态(MapState):将状态表示为一组Key-Value对
          \item 聚合状态:将状态表示为一个用于聚合操作的列表
      \end{itemize}
  
  \end{frame}

  \begin{frame}
      \frametitle{键控状态的使用}
  
      \begin{itemize}
          \item 声明一个键控状态
          \item 读取状态
          \item 写入状态
      \end{itemize}
  
  \end{frame}

  \begin{frame}
      \frametitle{状态后端(State Backends)}
  
      \begin{itemize}
          \item 每传入一条数据,有状态的算子任务都会读取和更新状态
          \item 由于有效的状态访问对于处理数据的低延迟至关重要,因此每个并行任务都会在本地维护其状态,以确保快速的状态访问
          \item 状态的存储、访问以及维护,由一个可插入的组件决定,这个组件就叫做状态后端(state backend)
          \item 状态后端主要负责两件事:本地的状态管理,以及将检查点(checkpoint)状态写入远程存储(HDFS、RocksDB之类的)
      \end{itemize}
  
  \end{frame}

  \begin{frame}
      \frametitle{选择一个状态后端}
  
      \begin{itemize}
          \item MemoryStateBackend(Default)
          \begin{itemize}
              \item 内存级的状态后端,会将键控状态作为内存中的对象进行管理,将它们存储在 TaskManager 的 JVM 堆上,而将 checkpoint 存储在 JobManager 的内存中
              \item 特点:快速、低延迟,但不稳定
          \end{itemize}
          \item FsStateBackend
          \begin{itemize}
              \item 将 checkpoint 存到远程的持久化文件系统(FileSystem)上,而对于本地状态,跟 MemoryStateBackend 一样,也会存在 TaskManager 的 JVM 堆上
              \item 同时拥有内存级的本地访问速度,和更好的容错保证              
          \end{itemize}
          \item RocksDBStateBackend
          \begin{itemize}
              \item 将所有状态序列化后,存入本地的 RocksDB 中存储。
              \item RocksDB是一个硬盘KV数据库,LevelDB,RocketDB
          \end{itemize}
      \end{itemize}
  
  \end{frame}

  \begin{frame}[plain,c]
    %\frametitle{A first slide}
    
    \begin{center}
    \Huge Q \& A
    \end{center}
    
  \end{frame}

\end{document}