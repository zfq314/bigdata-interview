\documentclass{beamer}
\usepackage{ctex}
\usepackage{minted}
\usepackage{smartdiagram}
\usetikzlibrary{positioning}
\usetheme{metropolis}           % Use metropolis theme
\title{Flink的容错机制}
\date{\today}
\author{左元}
\institute{尚硅谷 大数据组}
\begin{document}
  \maketitle
  \begin{frame}
    \frametitle{主要内容}

    \begin{itemize}
        \item 一致性检查点(checkpoint)
        \item 从检查点恢复状态
        \item Flink检查点算法(Chandy-Lamport算法的变种)
        \item 保存点(save points)
    \end{itemize}
  
  \end{frame}

  \begin{frame}
      \frametitle{一致性检查点(Checkpoints)}

      \begin{figure}
        \centering
        \includegraphics[width=0.6\textwidth]{image33.png}
        \caption{一致性检查点}
      \end{figure}
  
      \begin{itemize}
          \item Flink 故障恢复机制的核心,就是应用状态的一致性检查点
          \item 有状态流应用的一致检查点,其实就是所有任务的状态,在某个时间点的一份拷贝(一份快照)。
          \item 上游是一个可重置读取位置的持久化设备(Apache Kafka,txt文件)
      \end{itemize}
  
  \end{frame}

  \begin{frame}
      \frametitle{从检查点恢复状态}

      \begin{figure}
        \centering
        \includegraphics[width=0.6\textwidth]{image34.png}
        \caption{故障}
      \end{figure}
  
      \begin{itemize}
          \item 在执行流应用程序期间,Flink会定期保存状态的一致检查点
          \item 如果发生故障,Flink将会使用最近的检查点来一致恢复应用程序的状态,并重新启动处理流程
      \end{itemize}
  
  \end{frame}

  \begin{frame}
      \frametitle{从检查点恢复状态}

      \begin{figure}
        \centering
        \includegraphics[width=0.6\textwidth]{image35.png}
        \caption{重启应用}
      \end{figure}
  
      \begin{itemize}
          \item 遇到故障之后,第一步就是重启应用
      \end{itemize}
  
  \end{frame}

  \begin{frame}
      \frametitle{从检查点恢复状态}

      \begin{figure}
        \centering
        \includegraphics[width=0.6\textwidth]{image36.png}
        \caption{读取状态}
      \end{figure}
  
      \begin{itemize}
          \item 第二步是从checkpoint中读取状态,将状态重置
          \item 从检查点重新启动应用程序后,其内部状态与检查点完成时的状态完全相同
      \end{itemize}
  
  \end{frame}

  \begin{frame}
      \frametitle{从检查点恢复状态}

      \begin{figure}
        \centering
        \includegraphics[width=0.6\textwidth]{image37.png}
        \caption{重新消费}
      \end{figure}
  
      \begin{itemize}
          \item 第三步:开始消费并处理检查点到发生故障之间的所有数据
          \item 这种检查点的保存和恢复机制可以为应用程序状态提供“精确一次”(exactly-once)的一致性,因为所有算子都会保存检查点并恢复其所有状态,这样一来所有的输入流就都会被重置到检查点完成时的位置
      \end{itemize}
  
  \end{frame}

  \begin{frame}
      \frametitle{检查点的实现算法}
  
      \begin{itemize}
          \item 一种简单的想法(同步的思想)
          \begin{itemize}
              \item 暂停应用,保存状态到检查点,再重新恢复应用(Spark Streaming)
          \end{itemize}
          \item Flink 的改进实现(异步的思想)
          \begin{itemize}
              \item 基于Chandy-Lamport算法的分布式快照算法
              \item 将检查点的保存和数据处理分离开,不暂停整个应用
          \end{itemize}
      \end{itemize}
  
  \end{frame}

  \begin{frame}
      \frametitle{Flink检查点算法}
  
      \textbf{检查点分界线(Checkpoint Barrier,检查点屏障)}

      \begin{itemize}
          \item Flink 的检查点算法用到了一种称为分界线(barrier)的特殊数据形式,用来把一条流上数据按照不同的检查点分开
          \item 分界线之前到来的数据导致的状态更改,都会被包含在当前分界线所属的检查点中;而基于分界线之后的数据导致的所有更改,就会被包含在之后的检查点中
      \end{itemize}
  
  \end{frame}

  \begin{frame}
      \frametitle{Flink检查点算法}

      \begin{figure}
        \centering
        \includegraphics[width=0.6\textwidth]{image38.png}
        \caption{两个输入流的应用程序}
      \end{figure}
  
      \begin{itemize}
          \item 现在是一个有两个输入流的应用程序,用并行的两个Source任务来读取
      \end{itemize}
  
  \end{frame}

  \begin{frame}
      \frametitle{Flink检查点算法}

      \begin{figure}
        \centering
        \includegraphics[width=0.6\textwidth]{image39.png}
        \caption{启动检查点}
      \end{figure}
  
      \begin{itemize}
          \item JobManager会向每个Source任务发送一条带有新检查点ID的消息,通过这种方式来启动检查点
      \end{itemize}
  
  \end{frame}

  \begin{frame}
      \frametitle{Flink检查点算法}

      \begin{figure}
        \centering
        \includegraphics[width=0.6\textwidth]{image40.png}
        \caption{向下游传播检查点}
      \end{figure}
  
      \begin{itemize}
          \item 数据源将它们的状态写入检查点,并发出一个检查点barrier
          \item 状态后端在状态存入检查点之后,会返回通知给Source任务,Source任务就会向JobManager确认检查点完成
      \end{itemize}
  
  \end{frame}

  \begin{frame}
      \frametitle{Flink检查点算法}

      \begin{figure}
        \centering
        \includegraphics[width=0.6\textwidth]{image41.png}
        \caption{分界线对齐}
      \end{figure}
  
      \begin{itemize}
          \item 分界线对齐:barrier向下游传递,sum任务会等待所有输入分区的barrier到达
          \item 对于barrier已经到达的分区,继续到达的数据会被缓存
          \item 而barrier尚未到达的分区,数据会被正常处理
      \end{itemize}
  
  \end{frame}

  \begin{frame}
      \frametitle{Flink检查点算法}

      \begin{figure}
        \centering
        \includegraphics[width=0.6\textwidth]{image42.png}
        \caption{继续向下游转发检查点分界线}
      \end{figure}
  
      \begin{itemize}
          \item 当收到所有输入分区的barrier时,任务就将其状态保存到状态后端的检查点中,然后将barrier继续向下游转发
      \end{itemize}
  
  \end{frame}

  \begin{frame}
      \frametitle{Flink检查点算法}

      \begin{figure}
        \centering
        \includegraphics[width=0.6\textwidth]{image43.png}
        \caption{任务正常处理数据}
      \end{figure}
  
      \begin{itemize}
          \item 向下游转发检查点barrier后,任务继续正常的数据处理
      \end{itemize}
  
  \end{frame}

  \begin{frame}
      \frametitle{Flink检查点算法}

      \begin{figure}
        \centering
        \includegraphics[width=0.6\textwidth]{image44.png}
        \caption{检查点完成}
      \end{figure}
  
      \begin{itemize}
          \item Sink任务向JobManager确认状态保存到checkpoint完毕
          \item 当所有任务都确认已成功将状态保存到检查点时,检查点就真正完成了
      \end{itemize}
  
  \end{frame}

  \begin{frame}
      \frametitle{保存点(Savepoints)}
  
      \begin{itemize}
          \item Flink 还提供了可以自定义的镜像保存功能,就是保存点(savepoints)
          \item 原则上,创建保存点使用的算法与检查点完全相同,因此保存点可以认为就是具有一些额外元数据的检查点
          \item Flink不会自动创建保存点,因此用户(或者外部调度程序)必须明确地触发创建操作,savepoint是手动执行的
          \item 保存点是一个强大的功能。除了故障恢复外,保存点可以用于:有计划的手动备份,更新应用程序,版本迁移,暂停和重启应用,等等
      \end{itemize}
  
  \end{frame}

  \begin{frame}[plain,c]
    %\frametitle{A first slide}
    
    \begin{center}
    \Huge Q \& A
    \end{center}
    
  \end{frame}

\end{document}