\documentclass{beamer}
\usepackage{ctex}
\usepackage{minted}
\usepackage{smartdiagram}
\usepackage{diagbox}
\usetikzlibrary{positioning}
\usetheme{metropolis}           % Use metropolis theme
\title{Flink的状态一致性}
\date{\today}
\author{左元}
\institute{尚硅谷 大数据组}
\begin{document}
  \maketitle
  \begin{frame}
    \frametitle{主要内容}

    \begin{itemize}
        \item 状态一致性
        \item 一致性检查点(checkpoint)
        \item 端到端(end-to-end)状态一致性
        \item 端到端的精确一次(exactly-once)保证
        \item Flink+Kafka 端到端状态一致性的保证
    \end{itemize}
  
  \end{frame}

  \begin{frame}
      \frametitle{什么是状态一致性}
      
      \begin{figure}
      	\centering
      	\includegraphics[width=0.9\textwidth]{image34.png}
      	\caption{状态一致性}
      \end{figure}
  
      \begin{itemize}
          \item 有状态的流处理,内部每个算子任务都可以有自己的状态
          \item 对于流处理器内部来说,所谓的状态一致性,其实就是我们所说的计算结果要保证准确。
          \item 一条数据不应该丢失,也不应该重复计算
          \item 在遇到故障时可以恢复状态,恢复以后的重新计算,结果应该也是完全正确的。
      \end{itemize}
  
  \end{frame}

  \begin{frame}
      \frametitle{状态一致性分类}
  
      \begin{itemize}
          \item AT-MOST-ONCE(最多一次)
          \begin{itemize}
              \item 当任务故障时,最简单的做法是什么都不干,既不恢复丢失的状态,也不重播丢失的数据。At-most-once 语义的含义是最多处理一次事件。例如:UDP,不提供任何一致性保障
          \end{itemize}
          \item AT-LEAST-ONCE(至少一次)
          \begin{itemize}
              \item 在大多数的真实应用场景,我们希望不丢失事件。这种类型的保障称为 at-least-once,意思是所有的事件都得到了处理,而一些事件还可能被处理多次。
          \end{itemize}
          \item EXACTLY-ONCE(精确一次)
          \begin{itemize}
              \item 恰好处理一次是最严格的保证,也是最难实现的。恰好处理一次语义不仅仅意味着没有事件丢失,还意味着针对每一个数据,内部状态仅仅更新一次。
          \end{itemize}
      \end{itemize}
  
  \end{frame}

  \begin{frame}
      \frametitle{一致性检查点(Checkpoints)}
  
      \begin{itemize}
          \item Flink使用了一种轻量级快照机制 —— 检查点(checkpoint)来保证exactly-once语义
          \item 有状态流应用的一致检查点,其实就是:所有任务的状态,在某个时间点的一份拷贝(一份快照)。而这个时间点,应该是所有任务都恰好处理完一个相同的输入数据的时候(使用了检查点屏障)。
          \item 应用状态的一致检查点,是Flink故障恢复机制的核心
      \end{itemize}
  
  \end{frame}

  \begin{frame}
      \frametitle{一致性检查点(Checkpoints)}
      
      \begin{figure}
      	\centering
      	\includegraphics[width=0.9\textwidth]{image36.png}
      	\caption{一致性检查点}
      \end{figure}
  
  \end{frame}

  \begin{frame}
      \frametitle{端到端(end-to-end)状态一致性}
  
      \begin{itemize}
          \item 目前我们看到的一致性保证都是由流处理器实现的,也就是说都是在 Flink 流处理器内部保证的;而在真实应用中,流处理应用除了流处理器以外还包含了数据源(例如 Kafka)和输出到持久化系统
          \item 端到端的一致性保证,意味着结果的正确性贯穿了整个流处理应用的始终;每一个组件都保证了它自己的一致性
          \item 整个端到端的一致性级别取决于所有组件中一致性最弱的组件
      \end{itemize}
  
  \end{frame}

  \begin{frame}
      \frametitle{端到端Exactly-Once}
  
      \begin{itemize}
          \item 内部保证——checkpoint(分布式异步快照算法)
          \item Source端——可重设数据的读取位置(Kafka,FileSystem)
          \item Sink端——从故障恢复时,数据不会重复写入外部系统
          \begin{itemize}
              \item 幂等写入
              \item 事务写入
          \end{itemize}
      \end{itemize}
  
  \end{frame}

  \begin{frame}
      \frametitle{幂等写入(Idempotent Writes)}
  
      \textbf{所谓幂等操作,是说一个操作,可以重复执行很多次,但只导致一次结果更改,也就是说,后面再重复执行就不起作用了}
      
       \begin{figure}
      	\centering
      	\includegraphics[width=0.6\textwidth]{image45.png}
      	\caption{字典数据结构}
      \end{figure}
  
  \end{frame}

  \begin{frame}
      \frametitle{事务写入(Transactional Writes)}
  
      \begin{itemize}
          \item 事务(Transaction)
          \begin{itemize}
              \item 应用程序中一系列严密的操作,所有操作必须成功完成,否则在每个操作中所作的所有更改都会被撤消(ACID)
              \item 具有原子性:一个事务中的一系列的操作要么全部成功,要么一个都不做
          \end{itemize}
          \item 实现思想:构建的事务对应着checkpoint,等到checkpoint真正完成的时候,才把所有对应的结果写入Sink系统中
          \item 实现方式
          \begin{itemize}
              \item 预写日志(WAL,Write Ahead Log)(只能保证at-least-once)
              \item 两阶段提交(two-phase-commit、2PC)(可以保证exactly-once)
          \end{itemize}
      \end{itemize}
  
  \end{frame}

  \begin{frame}
      \frametitle{预写日志(Write-Ahead-Log,WAL)}
  
      \begin{itemize}
          \item 把结果数据(也就是要输出的数据)先缓存到状态后端,然后在收到 checkpoint 完成的通知时,一次性写入Sink系统(State Backend $\rightarrow$ MySQL)(万一写到中间的时候挂掉了呢?WAL只能保障at least once)
          \item 简单易于实现,由于数据提前在状态后端中做了存储,所以无论什么 sink 系统,都能用这种方式一批搞定
          \item DataStream API提供了一个模板类:GenericWriteAheadSink,来实现这种事务性Sink
      \end{itemize}
  
  \end{frame}

  \begin{frame}
      \frametitle{两阶段提交(Two-Phase-Commit,2PC)}
  
      \begin{itemize}
          \item 对于每个checkpoint,Sink任务会启动一个事务(下游设备的事务,比如MySQL,Kakfa),并将接下来所有接收的数据添加到事务里
          \item 然后将这些数据写入外部Sink系统,但不提交它们——这时只是“预提交”
          \item 当它收到checkpoint完成的通知时,它才正式提交事务,实现结果的真正写入
          \begin{itemize}
              \item 这种方式真正实现了Exactly-Once,它需要一个提供事务支持的外部Sink系统。Flink提供了TwoPhaseCommitSinkFunction接口。
              \item 有可能在一段时间内看不到Sink的结果
          \end{itemize}
      \end{itemize}
  
  \end{frame}

  \begin{frame}
      \frametitle{2PC对外部Sink系统的要求}
  
      \begin{itemize}
          \item 外部Sink系统必须提供事务支持,或者Sink任务必须能够模拟外部系统上的事务
          \item 在checkpoint的间隔期间里,必须能够开启一个事务并接受数据写入
          \item 在收到checkpoint完成的通知之前,事务必须是“等待提交”的状态。在故障恢复的情况下,这可能需要一些时间。如果这个时候Sink系统关闭事务(例如超时了),那么未提交的数据就会丢失
          \item Sink任务必须能够在进程失败后恢复事务
          \item 提交事务必须是幂等操作
      \end{itemize}
  
  \end{frame}

  \begin{frame}
      \frametitle{不同Sink的一致性保证}
  
      \begin{center}
          \begin{tabular}{|l|l|l|}
          \hline
          \diagbox[width=4cm]{sink}{source} & 不可重置的源 & 可重置的源 \\
          \hline
          any sink & at-most-once & at-least-once \\
          \hline
          幂等性sink & at-most-once & exactly-once \\
          \hline
          预写式日志sink & at-most-once & at-least-once \\
          \hline
          2PC sink & at-most-once & exactly-once \\
          \hline
          \end{tabular}%
      \end{center}
    
  \end{frame}

  \begin{frame}
      \frametitle{Flink+Kafka端到端状态一致性的保证}
  
      \begin{itemize}
          \item 内部 —— 利用 checkpoint 机制,把状态存盘(HDFS),发生故障的时候可以恢复,保证内部的状态一致性
          \item source —— kafka consumer 作为 source,可以将偏移量保存下来,如果后续任务出现了故障,恢复的时候可以由连接器重置偏移量,重新消费数据,保证一致性
          \item sink —— kafka producer 作为sink,采用两阶段提交 sink,需要实现一个 TwoPhaseCommitSinkFunction
      \end{itemize}
  
  \end{frame}

  \begin{frame}
      \frametitle{Exactly-Once两阶段提交}
      
      \begin{figure}
      	\centering
      	\includegraphics[width=0.7\textwidth]{image46.png}
      	\caption{两阶段提交}
      \end{figure}
  
      \begin{itemize}
          \item JobManager协调各个TaskManager进行checkpoint存储
          \item checkpoint保存在StateBackend中,默认StateBackend是内存级的,也可以改为文件级的进行持久化保存
      \end{itemize}
  
  \end{frame}

  \begin{frame}
      \frametitle{Exactly-Once两阶段提交}
      
      \begin{figure}
      	\centering
      	\includegraphics[width=0.7\textwidth]{image47.png}
      	\caption{两阶段提交}
      \end{figure}
  
      \begin{itemize}
          \item 当checkpoint启动时,JobManager会将检查点分界线(barrier)注入数据流
          \item barrier会在算子间传递下去
      \end{itemize}
  
  \end{frame}

  \begin{frame}
      \frametitle{Exactly-Once两阶段提交}
      
      \begin{figure}
      	\centering
      	\includegraphics[width=0.7\textwidth]{image48.png}
      	\caption{两阶段提交}
      \end{figure}
  
      \begin{itemize}
          \item 每个算子会对当前的状态做个快照,保存到状态后端
          \item checkpoint机制可以保证内部的状态一致性
      \end{itemize}
  
  \end{frame}

  \begin{frame}
      \frametitle{Exactly-Once两阶段提交}
      
      \begin{figure}
      	\centering
      	\includegraphics[width=0.7\textwidth]{image49.png}
      	\caption{两阶段提交}
      \end{figure}
  
      \begin{itemize}
          \item 每个内部的transform任务遇到barrier时,都会把状态存到checkpoint里
          \item Sink任务首先把数据写入外部Kafka,这些数据都属于预提交的事务;遇到barrier时,把状态保存到状态后端,并开启新的预提交事务
      \end{itemize}
  
  \end{frame}

  \begin{frame}
      \frametitle{Exactly-Once两阶段提交}
      
      \begin{figure}
      	\centering
      	\includegraphics[width=0.7\textwidth]{image50.png}
      	\caption{两阶段提交}
      \end{figure}
  
      \begin{itemize}
          \item 当所有算子任务的快照完成,也就是这次的checkpoint完成时,JobManager会向所有任务发通知,确认这次checkpoint完成
          \item Sink任务收到确认通知,正式提交之前的事务,Kafka中未确认数据改为“已确认”
      \end{itemize}
  
  \end{frame}

  \begin{frame}
      \frametitle{Exactly-Once两阶段提交步骤}
  
      \begin{itemize}
          \item 第一条数据来了之后,开启一个Kafka的事务(transaction),正常写入Kafka分区日志但标记为未提交,这就是“预提交”
          \item JobManager触发checkpoint操作,barrier从source开始向下传递,遇到barrier的算子将状态存入状态后端,并通知JobManager
          \item Sink连接器收到barrier,保存当前状态,存入checkpoint,通知JobManager,并开启下一阶段的事务,用于提交下个检查点的数据
          \item JobManager收到所有任务的通知,发出确认信息,表示checkpoint完成
          \item Sink任务收到JobManager的确认信息,正式提交这段时间的数据
          \item 外部Kafka关闭事务,提交的数据可以正常消费了
      \end{itemize}
  
  \end{frame}

  \begin{frame}[plain,c]
    %\frametitle{A first slide}
    
    \begin{center}
    \Huge Q \& A
    \end{center}
    
  \end{frame}

\end{document}